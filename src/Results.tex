% 	 1) Картинки с захватом и распределением по E,L
%		вначале и в конце эволюции (100 GeV, 100 KeV)
%	 2) Картинки для Скорости захвата и аннигиляции 
%		C(delta) и A(delta) с учётом аннигиляции и без %		учёта аннигиляции, сравнение с tan^2()
%	 3) Коэффициент аннигиляции a как функция от delta %		и m
%	 4) Графики в плоскости M,delta где есть равновесие	
%    
В нащем численном моделировании мы использовали неравномерную решетку: интервал по $l$ мы разбивали на $90$ равных частей, а по $E$ мы увеличивали плотность разбиения при мальеньких $E$.

Изобразим распределение частиц по энергии и моменту импульса (на единицу массы частицы). 

Мы будем рассматривать распределение частиц по энергтт и приведеному импульсу $f(E,l) = \frac{dN}{dE dl}$.

\begin{figure}[ht]
	\centering
	\includegraphics[width=0.7\textwidth]
	{images/Capt100_100.png}
	\caption{Распределение захваченных частиц для $m_{\chi} = 100\, \text{GeV}$, $\delta = 100\, \text{keV}$.}
	\label{fig:Capt100_100}
\end{figure}

\begin{figure}[!h]
	\centering
	\includegraphics[width=0.7\textwidth]
	{images/Dout100_100.png}
	\caption{Распределение частиц в конце эволюции для $m_{\chi} = 100\, \text{GeV}$, $\delta = 100\, \text{keV}$.}
	\label{fig:Dout100_100}
\end{figure}

\begin{figure}[!h]
\centering
\begin{tikzpicture}
	\begin{axis}[
		%title = {Распределение по $r/R_{\odot}$},
		no markers,
		xmin=0,
		xmax=0.2,
		xtick distance=0.02,
		 xticklabel style={
			/pgf/number format/fixed
		},
		xlabel = \(r\),
		ylabel = {\(\cfrac{d^3N}{d^3r}\)},
		table/col sep = semicolon,
		height = 0.2\paperheight, 
		width = 0.8\paperwidth,
		/pgf/number format/1000 sep={}
		]
		\addplot table [mark=none, x={R}, y={D100}]{tables/CaptDout100_100and50.csv};
		\addplot table [mark=none, x={R}, y={D50}]{tables/CaptDout100_100and50.csv};
		\addplot table [mark=none, x={R}, y={C100}]{tables/CaptDout100_100and50.csv};
		\addplot table [mark=none, x={R}, y={C50}]{tables/CaptDout100_100and50.csv};
		\addplot table [mark=none, x={R}, y={N}]{tables/Term100.csv};
		\legend{ 
			{$N(T_{\odot})$, $\delta = 100\text{keV}$},
			{$N(T_{\odot})$, $\delta = 50\text{keV}$},
			{$C$, $\delta = 100\text{keV}$},
			{$C$, $\delta = 50\text{keV}$},
			{$N_{therm}$}			
		};
	\end{axis}
\end{tikzpicture}
	\caption{Концентраця частиц тёмной материи массы $m_{\chi} = 100\text{GeV}$}
	\label{plot:Nrdistrib}
\end{figure}

Главный вопрос заключается в отношении скорости аннигиляции и скорости захвата $C/A$. Мы рассматриваем уравнение эволюции как с учётом аннигиляции так и без учета аннигиляции. 

\begin{figure}[!h]
	\centering
	\begin{tikzpicture}
		\begin{axis}[
			%title = {Распределение по $r/R_{\odot}$},
			no markers,
			xlabel = {$\delta$, $\text{keV}$},
			ylabel = {Rate, $\frac{1}{\text{s}}$},
			ymode=log,
			table/col sep = semicolon,
			height = 0.2\paperheight, 
			width = 0.8\paperwidth,
			/pgf/number format/1000 sep={}
			]
			\addplot table [mark=none, x={dM}, y={C}]{tables/CaptAnn_noAnn100.csv};
			
			\addplot table [mark=none, x={dM}, y={A}]{tables/CaptAnn_noAnn100.csv};
			
			\addplot table [mark=none, x={dM}, y={A}]{tables/Ann_withAnn100.csv};
			
			\addplot table [mark=none, x={dM}, y={Ath}]{tables/CaptAnn_noAnn100.csv};
			\legend{ 
				{Capture},
				{$A$, linear},
				{$A$, nonlinear},
				{$A = C \th^2{\sqrt{at^2C}}$}
			};
		\end{axis}
	\end{tikzpicture}
	\caption{Зависимость от $\delta$ захвата и аннигиляции при линейной и нелинейной эволюции для $m_{\chi} = 100\text{GeV}$}
	\label{plot:CaptAnn}
\end{figure}

Как можно заметить, если включить в уравнение эволюции аннигиляцию, то результат совпадает с оценкой (\ref{eq:AnnNtherm}). 


При больших $\delta$ уменьшается как величина захвата так и коэффициент аннигиляции. При $\delta$ ниже чем $\delta(m_{\chi})$, наступает термализация в уравнении (\ref{eq:Balance}), т.е. темп аннигиляции равен темпу захвата. Изобразим в плоскости $(m_{\chi} - \delta$ границы областей, при которых наступает или не наступает термализация при разных $\sigma_{\chi p}$. V


\begin{figure}[!h]
	\centering
	\begin{tikzpicture}
		\begin{axis}[
			%title = {Распределение по $r/R_{\odot}$},
			no markers,
			xlabel = {$\delta$, $\text{keV}$},
			ylabel = {M, $\text{GeV}$},
			ymode=log,
			table/col sep = semicolon,
			height = 0.3\paperheight, 
			width = 0.6\paperwidth,
			/pgf/number format/1000 sep={}
			]
			\addplot table [mark=none, x={d_40}, y={M}]{tables/DeltaM_SL.csv};
			\addplot table [mark=none, x={d_41}, y={M}]{tables/DeltaM_SL.csv};
			\addplot table [mark=none, x={d_42}, y={M}]{tables/DeltaM_SL.csv};
			\addplot table [mark=none, x={d_43}, y={M}]{tables/DeltaM_SL.csv};
			\addplot table [mark=none, x={d_44}, y={M}]{tables/DeltaM_SL.csv};
			\legend{ 
				{$\sigma_{\chi p} = 10^{-40}$},
				{$\sigma_{\chi p} = 10^{-41}$},
				{$\sigma_{\chi p} = 10^{-42}$},
				{$\sigma_{\chi p} = 10^{-43}$},
				{$\sigma_{\chi p} = 10^{-44}$}
			};
		\end{axis}
	\end{tikzpicture}
	\caption{}
	\label{plot:MDelta_SL}
\end{figure}



Обозначим величину $1/(aT_{\odot}^2)$ как $С_a(\avarage{\sigma_a v})$. Тогда для того чтобы сдлать ограничения для сечения на протоне $\sigma_{\chi p}$ исходя из ограничения на темп аннигиляции $\Gamma$ нужно решить (\ref{eq:AnnNtherm}) и найти темп захвата.

\begin{equation}
	\cfrac{2\Gamma}{C_a} = 
	F\left(\cfrac{C}{C_a}\right)
\end{equation}

где $F(x) = x\th^2{\sqrt(x)}$. Для оценки с точностью $0.06$ можно использовать $F^{-1}(y) \approx \sqrt{y(1+y)}$.

Тогда, зная $C/C_a$ и $C(\sigma_{\chi p,0})$ --- захват при $\sigma_{\chi p} = \sigma_{\chi p, 0}$ можно найти ограничение для сечения $\sigma_{\chi p}$
\begin{equation}
	\sigma_{\chi p} = \sigma_{\chi p, 0} \cfrac{C_a}{C(\sigma_{\chi p,0})} F^{-1}\left(	\cfrac{2\Gamma}{C_a}\right)
\end{equation}


