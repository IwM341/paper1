Когда тёмная материя захваттывается, она далее испытывает вторичные столкновения с веществом и аннигиляцию. Её эволюция описывается уравнением Больцмана. Для решения этого уравнения мы переходим из пространства $\vec{r}-\vec{v}$ в пространство $E-L$, где $E$ --- энергия тёмной материи, а $L$ --- момент импульса. Мы разбиваем пространство на бины, и тогда уравнение эволюции на число частиц $N_{i}$ в бине $i$ будет иметь вид:

\begin{equation}
	\label{eq:evolution}
	\deriv{N_{i}}{t} = \cfrac{1}{T_{\chi p}} \left(N_{\odot} c_i +
	\sum_j{[s_{ij} N_{j} - s_{ji} N_{i} ]} - e_{i} N_i - \cfrac{a_{\gamma}}{N_{\odot}} \sum_j {a_{ij} N_j N_i} \right)
\end{equation}

Величины $c_i$ определяются из интеграла (\ref{eq:ca_coeff}), но интеграл ограничивается областью, в которой конечные $E$ и $L$ частицы попадают в бин $i$.
Величины $s_{ij}$ определяют вероятности переёти из бина $j$ в бин $i$ и вычисляются аналогично $c_{i}$

\begin{equation}
	\label{eq:st_integral}
	s_{ij} = \sum_{\alpha}\int{
		d\Phi_{j}
		\cfrac{T_{in}(\Phi_{j})}{
			T_{in}(\Phi_{j})+T_{out}(\Phi_{j})
		} \cdot d\tau \left[ f_B(v_{\alpha}) d^3 \vec{v}_{\alpha} \right] \cdot \cfrac{v'}{v_{esc}} F(q,v) \cfrac{d\Omega}{4\pi}
	}
\end{equation}

\noindent Интеграл $d\Phi_{j}$ обозначает интегрирование по фазовому объёму бина $j$, причём полный интеграл $\int{d\Phi_{j}} = 1$ равен $1$. В выражении присутствует интегрирование вдоль траектории движения частицы в потенциале $\phi(r)$: $T_{in} + T_{out}$ --- время, за которое частицы проходит от минимального $r_{min}$ до максимального $r_{max}$ положения. Это время разбивается на время пребывания внутри Солнца $T_{in}$ и Снаружи $T_{out}$. Соответственно величина $\tau$ параметризует внутреннюю части траектории и пробегает от $0$ до $1$.

Те частицы, которые не попали ни в какой бин $i$ в интеграле (\ref{eq:st_integral}) испаряются, что определяет величину $e_i$.

Коэффициент $a_{\gamma}$ определяет соотношения скорости аннигиляции и столкновений и равен:

\begin{equation}
	a_{\gamma} = \cfrac{ \avarage{\sigma_{\chi\chi} v} n_{\chi 0.4}}{\sigma_{\chi p} n_{p} v_{esc}} = 
	T_{\chi p} \cdot \avarage{\sigma_{\chi\chi} v} n_{\chi 0.4}
\end{equation}
где $\avarage{\sigma_{\chi\chi} v}$ --- произведение сечения аннигиляции на скорость, которое мы брали равным $3\cdot10^{-26}\text{cm}^3\,{s^{-1}}$. Тогда $a_{\gamma} = 2.31\cdot10^{-16} \frac{\text{GeV}}{m_{\chi}}$. Выражения для $a_{ij}$ мы приведём в аппендиксе.

