

Захват тёмной материи происходит следующим образом. Изначально тёмная материя находится в гало (вдали от Солнца) и имеет некоторое распределение по скоростям $f(u)$. Затем она попадает в Солнце, где может столкнуться с ядрами вещества, потеряв часть кинетической энергии. Частица будет захвачена, если её финальная скорость будет меньше скорости вылета в точке столкновения $v’ < v_{esc}(r)$. Таким образом, величина захвата – это сумма по всем ядрам мишеням темпа столкновений тёмной материи:

\begin{equation}
	\label{eq:Capture_Full}
	C = \sum_{\alpha} C_{\alpha} = \sum_{\alpha} \int{
		d^3\vec{x} \cdot d^3\vec{v} \cdot f(v^2,r) \cdot n_{\alpha}(r) 
		d^3\vec{v}_{\alpha} f_B(v_{\alpha},T(r)) \cdot
		|\vec{v}- \vec{v}_{\alpha}| d\sigma_{\chi \alpha}
	}
\end{equation}

где $v$ --- скорость тёмной материи в точке $r$, $n_{\alpha}(r)$ --- концентрация ядер сорта $\alpha$, $f_B(v_{\alpha},T(r))$ --- Больцмановское распределение по скоростям ядер с локальной температурой $T(r)$, $d\sigma_{\chi \alpha}$ --- дифференциальное сечение тёмной материи и ядра. Интеграл по дифференциальному сечению идёт по области, когда $v’ < v_{esc}(r)$

Мы представим $C_{\alpha}$ следующим образом:

\begin{equation}
	\label{eq:Capture_a}
	C_{\alpha} = V_{\odot} n_{\chi 0.4} \cdot
	\sigma_{\chi p} n_{p} v_{esc} \cdot c_{\alpha} = 
	\cfrac{N_{\odot}}{T_{\chi p}} \cdot c_{\alpha}
\end{equation}

Где $V_{\odot}$ --- полный объем Солнца, $n_{\chi 0.4}$ --- локальная концентрация частиц тёмной материи ($0.4 \text{GeV}/\text{cm}^{3}$), $\sigma_{\chi p}$ --- характерное сечение взаимодействия тёмной материи и ядра, $v_{esc} = v_{esc}(r = R_{\odot})$ --- скорость вылета тёмной материи на краю Солнца. Мы определили $N_{\odot} = V_{\odot} n_{\chi 0.4} = 5.64\cdot10^{32} \frac{\text{GeV}}{m_{\chi}}$, а $T_{\chi p} = (\sigma_{\chi p} n_{p} v_{esc})^{-1}$, что равно $1.92\cdot 10^{10} \text{s}$.

Тогда 

\begin{equation}
	\label{eq:ca_coeff}
	c_{\alpha} = \int  \left[3r^2 dr  \right] \cdot \left[4\pi v \cdot u d u \cdot f_{eff}(u)  \right]  \cdot 
	\left[ \rho(r)\cfrac{\tilde{\rho}_{\alpha}(r)}{A_{\alpha}}  \cdot f_B(v_{\alpha}) d^3 \vec{v}_{\alpha} 
	\right] \cdot F(q,v) \cfrac{v'}{v_{esc}} \cdot \cfrac{d \Omega}{4\pi} 
\end{equation}

где: $r$ обозначает безразмерный радиус, меняющийся от $0$ до $1$, $\rho(r)$ --- плотность вещества, нормированная согласно $\int \rho(r) 3r^2 dr = 1$, $\tilde{\rho}_{\alpha}(r)$ --- массовая доля элемента $\alpha$ в точке $r$ (определяются из модели солнца AGSS09met \cite{SolarModel_2017}), $A_{\alpha}$ --- число нуклонов в ядре, $v'$ --- разность скоростей тёмной материи и ядра после столкновения. $f_{eff}(u)$ --- распределение частиц тёмной материи вдали от Солнца с учётом движения солнца со скоростью $u_0$ (скорость тёмной материи вдали Солнца $u$ и скорость $v$ в точке $r$ связанны соотношением $v^2 = u^2 + v_{esc}^2(r)$)

\begin{equation}
	f_e(u^2) = \int_{-1}^1 {f(u^2+u^2_0 + 2 u u_0 \cos{\theta}) \cfrac{d\cos{\theta}}{2}}
\end{equation}

В качестве распределения вдали от Солнца возьмём распределение Максвелла, ограниченное максимальной скоростью $u_{max}$

\begin{equation}
	f(u)  = \cfrac{1}{(2\pi \xi^2)^{3/2}} e^{-\frac{u^2}{2 \xi^2}} \cdot \theta(u_{max} - u )
\end{equation}

Мы предполагаем следующие параметры (из стандартной модели гало):
$v_{esc} = 2.06 \cdot 10^{-3} = 618 \frac{\text{km}}{\text{s}}$, $\xi = 0.423 \cdot 10^{-3} = 127 \frac{\text{km}}{\text{s}}$, $u_0 = 0.733 \cdot 10^{-3} = 220 \frac{\text{km}}{\text{s}}$, $u_{max} = 1.78 \cdot 10^{-3} = 533 \frac{\text{km}}{\text{s}}$.

$F(q,v)$ --- это форм фактор ядра, определённый следующим образом:

\begin{equation}
	\label{eq::ff_def}
	F(q,v) = \cfrac{(m_{\chi} + m_{p})^2}{(m_{\chi} + m_{\alpha})^2}
	\cfrac{|\mathcal{M}_{\chi \alpha}|^2}
	{|\mathcal{M}_{\chi p}|^2}
\end{equation}

где $|\mathcal{M}_{\chi \alpha}|^2$ --- усреднённый по спинам матричный элемент рассеяния тёмной материи на ядре, а $|\mathcal{M}_{\chi p}|^2$ --- на нуклоне.

Величину $\sigma_{\chi p}$ мы определяем как:

\begin{equation}
	\label{eq:sigma_def}
	\sigma_{\chi p} = \int {\cfrac{|\mathcal{M}_{\chi p}|^2}{64\pi^2 (m_{\chi} + m_p)^2} d\Omega}
\end{equation}



