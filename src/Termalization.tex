Когда тёмная материя захватывается, она далее испытывает вторичные столкновения с веществом и аннигиляцию. Её эволюция описывается уравнением Больцмана на функцию распределения $f(\vec{r},\vec{v})$:

\begin{equation}
	\label{eq:Boltsman}
	\deriv{f}{t}(\vec{r},\vec{v}) + \vec{v} \deriv{f}{\vec{r}}(\vec{r},\vec{v}) - \nabla \phi \deriv{f}{\vec{v}}(\vec{r},\vec{v}) = 
	St[f] (\vec{r},\vec{v})
\end{equation} 
где $St[f]$ --- интеграл столкновений. Этот член значительно меньше левой части уравнения (что следует из того, что длина свободного пробега частицы $\chi$ значительно превышает размер Солнца). Следовательно, для решения этого уравнения нужно перейти из пространства $\vec{r}-\vec{v}$ в пространство $E-L$, где $E$ --- энергия тёмной материи, а $L$ --- момент импульса. Линейная часть уравнение Больцмана, связанная с рассеяниям на ядрах, имеет вид:

\begin{equation}
	\label{eq:BoltsmanEL}
	\deriv{f}{t}(E,L) = C(E,L) + \int{ S(E_i,L_i,E,L) f(E_i,L_i) dE_idL_i } 
	-f(E,L) \int{ S(E,L,E_f,L_f)  dE_fdL_f }
\end{equation}

В уравнение входят также член, отвечающий за аннигиляцию, и член, отвечающий за рассеяние частиц тёмной материи на самой себе, который в данной работе считается малым по сравнениею с аннигиляцией и не учитывается.

Величины $C(E,L)$ и $S(E,L,E',L')$ вычисляются как интегралы столкновений (\ref{eq:ca_coeff}) и (\ref{eq:st_integral}) но с добавлением $\delta$-функций по $E$ и $L$.

Для учёта аннигиляции в (\ref{eq:BoltsmanEL}) добавляется 
\begin{equation}
	-\label{eq:BoltsmanAnn}
	f(E,L) \int{ A(E,L,E',L') f(E',L') dE'dL' }
\end{equation}

Для удобства, мы используем безразмерные величины энергии $E$ и момента импульса $L$, которые определены следующим образом:
\begin{equation}
	\begin{split}
		E = \cfrac{\frac{1}{2} v_{\chi}^2 + \phi(r)}
		{\frac{1}{2} v_{esc}^2} \\
		L = \cfrac{|\vec{r} \times \vec{v}|}
		{R_{\odot} v_{esc}}
	\end{split}
	\label{eq:dimentionless}
\end{equation}

Еще одна величина --- приведенный момент импульса.
\begin{equation}
	l = \cfrac{L}{L_{max}(E)}
\end{equation}

где $L_{max}(E)$ --- максимально возможный момент импулься траектории при данной энергии (при этом не учитываются траектории, которые находятся вне Солнца).



Для численного решения уравнения мы разбиваем пространство на прямоугольные бины по величинам энергии $E$ и приведенного момента $l$. Внутри бина мы считаем, что частицы распределены относительно какой-то меры: это может быть $d\Phi = dEdL$ либо $d\Phi = dEdL^2$ (мы используем второй вариант). 

Eравнение эволюции на число частиц $N_{i}$ в бине $i$ будет иметь вид:

\begin{equation}
	\label{eq:evolution}
	\deriv{N_{i}}{t} = \cfrac{1}{T_{\chi p}} \left(N_{\odot} c_i +
	\sum_j{[s_{ij} N_{j} - s_{ji} N_{i} ]} - e_{i} N_i - \cfrac{a_{\gamma}}{N_{\odot}} \sum_j {a_{ij} N_j N_i} \right)
\end{equation}

Величины $c_i$ определяются из интеграла (\ref{eq:ca_coeff}), но интеграл ограничивается областью, в которой конечные $E$ и $L$ частицы попадают в бин $i$.
Величины $s_{ij}$ определяют вероятности переёти из бина $j$ в бин $i$ и вычисляются аналогично $c_{i}$. Как $c_{i}$ так  $s_{ij}$ мы вычисляем методом Монте-Карло.

\begin{equation}
	\label{eq:st_integral}
	s_{ij} = \sum_{\alpha}\int{
		\cfrac{d\Phi_{j}}{\mu(\Phi_{j})}
		\cfrac{T_{in}(\Phi_{j})}{
			T_{in}(\Phi_{j})+T_{out}(\Phi_{j})
		} \cdot d\tau \left[ f_B(v_{\alpha}) d^3 \vec{v}_{\alpha} \right] \cdot \cfrac{v'}{v_{esc}} F(q,v) \cfrac{d\Omega}{4\pi}
	}
\end{equation}

\noindent Интеграл $d\Phi_{j}$ обозначает интегрирование по фазовому объёму бина $j$, причём $\mu(\Phi_{j}) = \int{d\Phi_{j}}$ равен $1$. В выражении присутствует интегрирование вдоль траектории движения частицы в потенциале $\phi(r)$. Величина $\tau$ параметризует внутреннюю части траектории и пробегает от $0$ до $1$. $T_{in} + T_{out}$ --- время, за которое частицы проходит от минимального $r_{min}$ до максимального $r_{max}$ положения. Это время разбивается на время пребывания внутри Солнца $T_{in}$ и Снаружи $T_{out}$. Мы используем также обезразмеренное время на $R_{\odot}/v_{esc}$.

\begin{equation*}
	T(E,L) = \int_{r_{min}}^{r_{max}}{\cfrac{dr}{
			\sqrt{E - \varphi(r) - \frac{L^2}{r^2}}
		} 
	}
\end{equation*}

где $\varphi(r)$ --- определяет Ньютоновский потенциал:

\begin{equation*}
	\phi(r) = \cfrac{GM(r)}{r} = \varphi(r) \cfrac{v_{esc}^2}{2} 
\end{equation*}

$\varphi(r)$ удовлетворяет уравнению Пуассона:

\begin{equation}
	\cfrac{1}{r^2}\deriv{}{r}
	\left(r^2\deriv{\varphi(r)}{r}\right) = 3\rho(r)
\end{equation}

где $\rho(r)$ --- отношение плотности к средней плотности из (\ref{eq:ca_coeff}).


Те частицы, которые не попали ни в какой бин $i$ в интеграле (\ref{eq:st_integral}) испаряются, что определяет величину $e_i$, т.е. $e_i$ = $s_{i'i}$, где $i'$ --- формальный индекс бина: $(E,l) \in [0,+\infty]\times[0,1]$.

Коэффициент $a_{\gamma}$ определяет соотношения скорости аннигиляции и столкновений и равен:

\begin{equation}
	a_{\gamma} = \cfrac{ \avarage{\sigma_{\chi\chi} v} n_{\chi 0.4}}{\sigma_{\chi p} n_{p} v_{esc}} = 
	T_{\chi p} \cdot \avarage{\sigma_{\chi\chi} v} n_{\chi 0.4}
\end{equation}
где $\avarage{\sigma_{\chi\chi} v}$ --- произведение сечения аннигиляции на скорость, которое мы брали равным $3\cdot10^{-26}\text{cm}^3\,{s^{-1}}$. Тогда $a_{\gamma} = 2.31\cdot10^{-16} \frac{\text{GeV}}{m_{\chi}}$. Выражения для $a_{ij}$ мы приведём в аппендиксе.

