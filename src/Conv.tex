%	Сходимость схем для упругого случая
%	Сходимость схемы для неупругого случая
%	Собственные значения матрицы рассеяния+ 
%   независимость результатов от шага по времени.

Для проверки точности численных схем, мы рассмотрели эволюцию в упругом и неупругом случае.

В упругом случае мы увеличивали число бинов по энергии и смотрели, как зависит аннигиляция от числа бинов $N_E$ после термализации. 

Хотя наивно схема имеет первый порядок точности, на самом деле порядок может быть и вторым.

Рассмотрим уравнение эволюции:

\begin{equation}
	\deriv{f(X)}{t} = \int{S(X,X')f(X')dX'} - 	 f(X)\int{S(X',X)dX'} 
\end{equation}

Пусть $P_h$ --- проектор решения на сетку. Тогда
\begin{equation}
	P_h\deriv{f(X)}{t} = P_h\int{S(X,X')f(X')dX'} - 	 P_h f(X)\int{S(X',X)dX'} 
\end{equation}

\begin{equation}
	\deriv{f_h(X)}{t} = \mu(dX)^{-1} \int{S(X,X')[f(X')-f(X)]dX'dX}
\end{equation}

После дискретизации оператор эволюции $S$ равен 
\begin{equation}
	S_h(X,X') = \mu(dX)^{-1} \mu(dX')^{-1} \int{S(X,X')dXdX'}
\end{equation}

Положим $S(X,X') = S(X_0,X_0') + A\Delta X + B \Delta X') + O(h^2)$. Из чего следует, что $S_h(X,X')$ = $S(X_0,X_0')$. 

Далее положим, что $f(X) = f_h(X) + F_1\Delta X$

Тогда 

\begin{equation}
	\deriv{f_h(X)}{t} = \mu(dX)^{-1} \int{S_h(X,X')[f_h(X')-f_h(X)]dX'dX} + O(h^2) 
\end{equation}

Получаем, что если $S(X,X')$ и $f(X)$ --- гладкие, то схема имеет 2 порядок точности по $E,L$.



Для проверки сходимости в упругом случае мы нашли конечное распределение частиц тёмной материи при разных шагах решетки и вычислили среднюю энергию и темп аннигиляции на этих распределениях. В упругом случае мы рассматривали зависимость от шага по энергии $h_e$, который мы уменьшали в 2 раза. Поскольку термальное распределение по моменту импульса не зависит от $L$ и схема по $h_L$ не имеет ошибки, мы рассматривали зависимость только от $h_e$. 



\begin{figure}[!h]
	\centering
	\begin{tikzpicture}
		\begin{axis}[
			at={(0,0)}, % Позиция (можно регулировать)
			width=0.5\textwidth,
			height=0.4\textwidth,
			table/col sep = semicolon,
			xmin=0,
			xlabel = { \(h_e/{T_{\odot}} \)},
			ylabel = {\(\avarage{E}/\avarage{E_{1/8}}\)}
			]
			\addplot table [only marks, scatter,x={he}, y={E}]{tables/Conv10.csv};
			\addplot [no marks,domain=0:1.7] {0.999 + 0.02*x^2};
		\end{axis}
		\begin{axis}[
			at={(0.5\textwidth,0)}, % Позиция (можно регулировать)
			width=0.5\textwidth,
			height=0.4\textwidth,
			table/col sep = semicolon,
			xmin=0,
			xlabel = { \(h_e/{T_{\odot}} \)},
			ylabel = {\(A/A_{1/8}\)}
			]
			\addplot table [x={he}, y={A}]{tables/Conv10.csv};
			\addplot [no marks,domain=0:1.7] {1.003 - 0.048*x^2};
		\end{axis}
	\end{tikzpicture}
	\caption{Зависимость средней энергии $\avarage{E}$ и аннигиляции $A$ при $m = 10 \text{GeV}$. Шаг по энергии указан в единицах температуры Солнца в центре $T_{\odot}$. Значения нормированны на значениях при минимальном шаге.}
	\label{plot:ConvEl10}
\end{figure}

В неупругом случае мы рассматривали сетки с разбиением по $E$ и $L$ на $50$,$100$ и $200$ равномерных отрезков.




\begin{figure}[!h]
	\centering
	\begin{tikzpicture}
		\begin{axis}[
			at={(0,0)}, % Позиция (можно регулировать)
			width=0.5\textwidth,
			height=0.4\textwidth,
			table/col sep = semicolon,
			xmin=0,
			xlabel = { \(h_e (1 = E_{max}/50) \)},
			ylabel = {\(A/A_{1/4,1/4}\)},
			legend style={
				at = {(0.3,0.8)},
				anchor=north
			}
			]
			\addplot table [x={h}, y={nl0}]{tables/ConvInel_E.csv};
			\addplot table [x={h}, y={nl1}]{tables/ConvInel_E.csv};
			\addplot table [x={h}, y={nl2}]{tables/ConvInel_E.csv};
			\legend{ 
				{$h_L = 1\ \ \ $},
				{$h_L = 1/2$},
				{$h_L = 1/4$}
			};
		\end{axis}
		\begin{axis}[
			at={(0.5\textwidth,0)}, % Позиция (можно регулировать)
			width=0.5\textwidth,
			height=0.4\textwidth,
			table/col sep = semicolon,
			xmin=0,
			xlabel = { \(h_L (1 = L_{max}/50) \)},
			ylabel = {\(A/A_{1/4,1/4}\)},
			legend style={
				at = {(0.2,0.98)},
				anchor=north
			}
			]
			\addplot table [x={h}, y={ne0}]{tables/ConvInel_L.csv};
			\addplot table [x={h}, y={ne1}]{tables/ConvInel_L.csv};
			\addplot table [x={h}, y={ne2}]{tables/ConvInel_L.csv};
			\legend{ 
				{$h_e = 1\ \ \ $},
				{$h_e = 1/2$},
				{$h_e = 1/4$}
			};
		\end{axis}
	\end{tikzpicture}
	\caption{Зависимость средней энергии $\avarage{E}$ и аннигиляции $A$ при $m = 10 \text{GeV}$. Шаг по энергии указан в единицах температуры Солнца в центре $T_{\odot}$. Значения нормированны на значениях при минимальном шаге. $m_{\chi} = 100 \text{GeV}$, $\delta= 100 \text{keV}$}
	\label{plot:ConvInel100}
\end{figure}

