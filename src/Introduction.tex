Наличие тёмной материи – это хорошо хорошо установленный факт, следующий из астрофизических наблюдений и космологических данных. Тёмная материя составляет основную массу галактик, объясняя форму кривой вращения, входит в уравнения Фридмана эволюции Вселенной а также играет ключевую роль в образовании крупномасштабной структуры Вселенной. 	Тёмная материя практически не взаимодействует с веществом, поэтому состав тёмной материи неизвестен. Вероятно, что она состоит из новых частиц вне Стандартной модели физики частиц. Одним из классов таких частиц являются слабовзаимодействующие массивные частицы (WIMPs), которые мы будем рассматривать.


Образование частиц тёмной материи объясняется механизмом freeze-out, при котором частицы находились в термальном равновесии с  плазмой, о потом вышли из равновесия вследствие расширения Вселенной. Для объяснения плотности тёмной материи $\Omega h^2 = 0.12$ необходимо, чтобы сечение аннигиляции тёмной материи в частицы стандартной модели имело порядок $10^{-26} \text{см}^3/\text{c}$. Сечения такого порядка характерны для процессов электрослабого масштаба. В таком случае эти частицы можно зарегистрировать различными способами. Наиболее чувствительным методом является прямое детектирование частиц в галактическом гало путём измерения отдачи ядер в низкофоновых экспериментах, таких как XENON, PANDAX. Отсутствие сигнала накладывает сильные ограничения на упругое сечение взаимодействия с нуклоном $\sigma_{\chi n}$ (рисунок).


Однако эти ограничения могут быть ослаблены, если тёмная материя является неупругой: частицам в гало не будет хватать энергии для преодоления порога реакции рассеяния на ядре. Тогда более чувствительным методом поиска тёмной материи может стать детектирование нейтрино от аннигиляции тёмной материи захваченной и накопленной внутри Солнца. Для того, чтобы дать предсказание потоков нейтрино нужно найти скорость захвата частиц тёмной материи $C$, распределение этих частиц на текущий момент и темп аннигиляции. Имея темп аннигиляции можно найти спектр нейтрино. В случае упругой тёмной материи частицы быстро термализуются и имеют распределение Больцмана с температурой близкой к температуре в центре Солнца. Для исследуемой области параметров масса - сечение $m_{\chi} - \sigma{\chi p}$ это означает, что темп аннигиляции в текущий момент должен совпадать с темпом захвата. Однако как было показано в работе \cite{Distribution_2018}, неупругая тёмная материя в какой-то момент времени перестанет термализоваться из-за энергетического порога. В таком случае темп аннигиляции может быть существенно меньше темпа захвата, что уменьшит чувствительность метода и, значит, ослабит ограничения.


В данной работе мы рассмотрим численное решение термализации неупругой тёмной материи в Солнце и найдём соотношение между аннигиляцией и захватом а также найдём область параметров тёмной материи, при которой равновесие между аннигиляцией и захватом всё ещё наступает. Мы будем предполагать, что время распада возбужденного состояния значительно меньше времени жизни Вселенной и в гало возбужденое состояние отсутствует.