Для того, чтобы была аннигиляция тёмной материи в Солнце, она должна накопиться в достаточном количестве. Количество частиц, захваченное из гало за единицу времени обозначают как $C$. Темп аннигилляции $A$ зависит от количества частиц $N$ и их распределения. Темп захвата равен $aN^2$, где $a$ --- зависит только от распределения частиц. 

Уравнение на полное число частиц выглядит следующим образом:

\begin{equation}
	\tderiv{N} = C - aN^2
\end{equation}

\noindent и имеет решение в момент времени $t$:
\begin{equation}
\begin{split}
	N = \sqrt{\cfrac{C}{a}} \th{[\sqrt{aC}t]} \\
	A = aN^2 = C \th^2{[\sqrt{at^2C}]}
	\label{eq:AnnNtherm}
\end{split}
\end{equation}
	

В случае упругой тёмной материи коэффициент $a$ определяется термальным распределением:
\begin{equation}
	a_{el} = \cfrac{
		\avarage{\sigma_{a}v} \int{d^3 r e^{-2 \frac{m_{\chi}\phi(r)}{T_{\chi}} }}
	}{\left(
		\int{d^3 r e^{- \frac{m_{\chi}\phi(r)}{T_{\chi}} }}
	\right)^2}
\end{equation}

В случае, когда масса тёмной материи больше $10$ GeV, температуру термального распределения $T_{\chi}$ можно взять равной температуре в центре Солнца $T_{\odot}(r=0) = 1.33\cdot10^{-6} \text{GeV}$, а потенциал $\phi(r)$ разложить до квадратичного порядка: $\phi(r) =\phi(0)+ \frac{v_{esc}^2}{2} \frac{\rho_0}{2} \frac{r^2}{R_{\odot}^2} + o(r^2)$,  где $\rho_0 = 105.7$. Тогда

\begin{equation}
	a_{el} = \cfrac{1}{(4\pi)^{3/2}} \cfrac{\avarage{\sigma_{a}v}}{ r_{\chi}^3 }
\end{equation}

где

\begin{equation}
	r_{\chi} = R_{\odot} \sqrt{\cfrac{2 T_{\odot}(0) }{m_{\chi}\rho_0 v_{esc}^2}} =  0.077 R_{\odot} \sqrt{\cfrac{ \text{GeV}}{m_{\chi}}}
\end{equation}

Приведём значение для $a_{el}t^2$ в момент возраста Солнца $T_{\odot}$

\begin{equation}
	a_{el}T_{\odot}^2 = 
	9\cdot10^{-23} \text{s}	\left(\cfrac{   \avarage{\sigma_{a}v}    }{3\cdot10^{-26} \text{cm}^2\text{s}^{-1}} \right)
	\left(\cfrac{m_{\chi}}{\text{GeV}}\right)^{3/2}
\end{equation}

Для неупругого случая, уравнения эволюции хоть и не учитывают распределение частиц, но всё равно оказываются верными в переходной области, где $C$ и $A$ имеют схожее значение. 
В неупругом случае этот зависит от массы тёмной материи $m_{\chi}$, разницей масс основного и возбуждеённого состояния $\delta$ и от сечения аннигиляции $\avarage{\sigma_{a}v}$ (мы брали $\avarage{\sigma_{a}v} = 3\cdot 10^{-26} \text{cm}^3/\text{s}$). Поэтому для нахождения темпа аннигиляции нужно воспользоваться формулой выше и наёденными численно коэффицентами $a$.